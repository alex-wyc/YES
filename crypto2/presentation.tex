\documentclass{beamer}
\usepackage{amssymb}
\usepackage{listings}
\usetheme{Berlin}
\usecolortheme{beaver}
\usefonttheme{structuresmallcapsserif}

\AtBeginSection[]
{
    \begin{frame}
      \frametitle{Table of Contents}
      \tableofcontents[currentsection]
    \end{frame}
}

\title{Introduction to Cryptography 2}
\subtitle{Polyalphabetic Substitution Ciphers \# 1}
\author{Yicheng Wang}
\institute{White Hat Academy}
\date{2015-05-04}
\subject{Computer Science}

\definecolor{codegreen}{rgb}{0,0.6,0}
\definecolor{codegray}{rgb}{0.5,0.5,0.5}
\definecolor{codeblue}{rgb}{0,0,0.6}
\definecolor{backcolor}{rgb}{0.95,0.95,0.95}

\lstdefinestyle{mystyle}{
	backgroundcolor = \color{backcolor},
	commentstyle = \color{codeblue},
	keywordstyle = \color{codegreen},
	numberstyle = \color{codegray},
	stringstyle = \color{magenta},
	basicstyle = \tiny\ttfamily,
	breakatwhitespace = false,
	breaklines = true,
	captionpos = b,
	keepspaces = true,
	numbers = left,
	numbersep = 5pt,
	showspaces = false,
	showstringspaces = false,
	showtabs = false,
	tabsize = 4
}

\lstset{style = mystyle}


\begin{document}

\frame{\titlepage}

\begin{frame}
    \frametitle{Review}
    Decipher this:

    \begin{center}
    Zgyzhs rh vzhb
    \end{center}

    This is encrypted using a monoalphabetic substitution cipher.
\end{frame}

\begin{frame}
    \frametitle{Polyalphabetic Substitution Cipher}
    \begin{itemize}
        \item As we've seen last time. Monoalphabetic substitution ciphers are
            not the most secure way of encrypting things.
        \item To get around this, people have invented polyalphabetic
            substitution cipher, this uses a different substitution system for
            each letter.
        \item Today we'll look at two basic polyalphabetic substitution
            ciphers: Tabula Recta and Vigenère Cipher.
    \end{itemize}
\end{frame}

\begin{frame}
    \frametitle{Tabula Recta}
    \begin{itemize}
        \item This cipher uses the table as illustrated on the next slide.
        \item It has 26 rows, each row a shift of the alphabet.
        \item With each letter we use a different row to encrypt it. For
            example, the tabula recta encryption of ``hello'' is ``igopt.''
    \end{itemize}
\end{frame}

\begin{frame}
    \begin{center}
        \begin{figure}[h]
            \includegraphics[width=0.6\textwidth]{Vigenère_square_shading}
            \caption{The Tabula Recta Table}
        \end{figure}
    \end{center}
\end{frame}

\begin{frame}[fragile]
\begin{lstlisting}[language=java]
public class TabulaRecta {
    private char[][] board;
    private final String alphabet = "abcdefghijklmnopqrstuvwxyz";
    private char[] plainArray;

    private void genBoard() {
        board = new char[alphabet.length()][alphabet.length()];
        for (int i = 0 ; i < alphabet.length() ; i++) {
            for (int j = 0 ; j < alphabet.length() ; j++) {
                board[i][j] = alphabet.charAt((j + i) % 26);
            }
        }
    }
    public TabulaRecta(String s) {
        genBoard();
        plainArray = s.toCharArray();
    }
    public String encrypt() {
        StringBuilder cipherText = new StringBuilder();
        for (int i = 0 ; i < plainArray.length ; i++) {
            cipherText.append(board[(i + 1) % board.length][indexOf(board[0], plainArray[i])]);
        }
        return cipherText.toString();
    }
    public int indexOf(char[] arr, char el) {
        for (int i = 0 ; i < arr.length ; i++) {
            if (arr[i] == el) {
                return i;
            }
        }
        return -1;
    }
}
\end{lstlisting}
\end{frame}

\begin{frame}
\frametitle{Vigenère Cipher}
    \begin{itemize}
        \item Vigenère cipher is an improvement upon the old Tabula Recta
            cipher.
        \item It is developped by French cryptographer Blaise de Vigenère.
        \item It uses a key to determine which row to go to instead of going
            throw the entire alphabet from A to Z.
        \item For example, if the message is ``APCSTESTTHURSDAY'' and the key
            being ``JAVA'', we would first repeat the key until it matchs the
            length of the message, which would make the key ``JAVAJAVAJAVAJAVA.''
        \item Then, for each letter is encrypted using the tabula recta method
            on the row of the corresponding letter in the key.
        \item So ``APCSTESTTHURSDAY'' will encrypt to ``JPXSCENTCHPRBDVY''
    \end{itemize}
\end{frame}

\begin{frame}
\frametitle{Breaking Vigenère}
    \begin{itemize}
        \item The idea behind all polyalphabetic substitution ciphers is to
            disguise letter frequency to disrupt regular frequency analysis.
        \item If we have a crib, this makes our lives a lot easier. Because with
            the crib, the ciphertext and the encryption table, we can easily
            figure out the table and find the key, which in turn allows us to
            find the plaintext.
        \item Here's a contrieved exercise, we have intercepted the following
            encrypted message:

            \begin{center}
                mtltgbfoehviawsmhrin
            \end{center}

        \item We know that the plaintext starts with the word ``attack.'' and
            the key (hopefully) also has that length.
        \item Figure out the original message!
    \end{itemize}
\end{frame}

\begin{frame}
    \begin{itemize}
        \item However, we don't always know a crib, or the key length for that
            matter. Once we have the keylength, it is easy to figure out
            where the key repeats and Vigenere cipher breaks down to a series of
            interwoven Caesar shifts, each of which can be easily brute forced.
        \item The keylengths are determined by the Kasiski and Friedman tests.
            Which will be covered, eventually. (Mathy stuff ahead)
    \end{itemize}
\end{frame}
\end{document}
